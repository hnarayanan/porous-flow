\section{Formulation}
\label{formulation}

A detailed formulation will be described here. \cite{allen:1985} is a
classical reference describing the mathematical model.

\subsection{Steady-state Darcy flow}
\label{darcy-flow}

This program solves pressure-driven, steady-state Darcy's flow on a
square plate with spatially varying permeability.

\begin{equation}
\begin{split}
\bK^{-1}\bu + \grad(p) & = 0\\
\div(\bu) & = 0
\end{split}
\end{equation}

Which, in weak form reads:

\begin{equation}
\begin{split}
(\bv, \bK^{-1}\bu) - (\div(\bv), p) = & - (\bv, p\bn)_N \quad\forall v \\
(q, \div(\bu)) = & 0 \quad\forall q
\end{split}
\end{equation}

It then proceeds to construct and solve the adjoint problem, with a
suitable goal functional of interest as its right hand-side.

\begin{equation}
(\bw, \bK^{-1}\bv) - (\div(\bw), q) + (r, \div(\bv)) = M(v) \quad\forall
v, q
\end{equation}

We solve this adjoint problem for the variable \bz{} = (\bw, r). After
this, the error is estimated (rather crudely for now: ignoring jump
terms and constant multipliers) using the dual weighted residual
method:

\begin{equation}
M(\bu) - M(\bu_h) \approx  \sum_T |<\bR_T, \bz - \bz_h>_T + <\bR_{\partial T}, \bz - \bz_h>_{\partial T}|
\end{equation}

The error estimate is used to suitably refine the mesh, and the above
process is repeated until a certain tolerance is reached.

\subsection{Two-phase flow}
\label{two-phase-flow}

This program solves pressure-driven, time-dependent flow of two phases
through porous media.

{\bf Strong form:}

\begin{equation}
\begin{split}
  (\lambda(s)\bK)^{-1}\bu + \grad(p) & = 0\\
  \div(\bu) & = 0\\
  \frac{\partial s}{\partial t} + \bu\cdot\grad(F(s)) & = 0,
\end{split}
\end{equation}

where,

\begin{equation}
\begin{split}
  \lambda(s) = & 1.0/\mu_{\mathrm{rel}} s^2 + (1 - s)^2 \\
        F(s) = & k_{rw}(s)/\mu_w/(k_{rw}(s)/\mu_w + k_{ro}(s)/\mu_o) \\
             = & s^2/(s^2 + \mu_{\mathrm{rel}}(1 - s)^2).
\end{split}
\end{equation}

One can then can post-calculate the velocity of each phase using the
relation: $\bu_j = - (k_{rj}(s)/\mu_j)\bK\grad(p)$.

{\bf Weak form:}

Find $\bu, p, s \in V$ such that,

\begin{equation}
\begin{split}
(\bv, (\lambda\bK)^{-1}\bu) - (\div(v), p) = & - (\bv, \bar{p}\bn)_N \\
                           (q, \div(\bu)) = & 0\\
(r, \frac{\partial s}{\partial t}) - (\grad(r), F\bu) = & - (r,
F\bu\cdot\bn)_N
\end{split}
\end{equation}

$\forall \bv, q, r \in \hat{V}.$

{\bf Adjoint problem:}

Find $\bz_u, z_p, z_s in \hat{V}$ such that,

\begin{equation}
(\bz_u, (\lambda\bK)^{-1}\bv) - (\div(\bz_u), q) + (z_p, \div(\bv))
+ (z_s, r) - (z_s, s0) - dt (\grad(z_s), F\bv) + dt (z_s,
F\bv\cdot\bn)_N = M(\bv)
\end{equation}

$\forall \bv, q, r \in V - V_h.$

{\bf Error estimate (per time step):}

Using the dual weighted residual method (ignoring jump terms and and
constant multipliers),

\begin{equation}
M(\bu) - M(\bu_h) \approx  \sum_T |<\bR_T, \bz - \bz_h>_T +
<\bR_{\partial T}, \bz - \bz_h>_{\partial T}| .
\end{equation}

% Model problem:

%  -----4-----
%  |         |
%  1         2
%  |         |
%  -----3-----

% Initial conditions:
% u(x, 0) = 0
% p(x, 0) = 0
% s(x, 0) = 0 in \Omega

% Boundary conditions:
% p(x, t) = 1 - x on \Gamma_{1, 2, 3, 4}
% s(x, t) = 1 on \Gamma_1 if u.n < 0
% s(x, t) = 0 on \Gamma_{2, 3, 4} if u.n > 0

% Goal functionals:
% M(v) = inner(grad(u_h), grad(v))*dx
% M(v) = inner(u_h, v)*dx
% M(v) = inner(v, n)*ds(2)

% Parameters:
% mu_rel, Kinv, lmbdainv, F, dt, T

% This implementation includes functional forms from the deal.II demo
% available at: http://www.dealii.org/6.2.1/doxygen/deal.II/step_21.html

% Local Variables:
% TeX-master: "adaptive-porous-flow"
% mode: latex
% mode: flyspell
% End:
